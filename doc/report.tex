\documentclass[a4paper,11pt]{article}

\usepackage[utf8]{inputenc}
%\usepackage[only,llbracket,rrbracket]{stmaryrd}
\usepackage{a4wide}
\usepackage[a4paper]{geometry}
\usepackage{mathtools}
\usepackage[T1]{fontenc}
\usepackage{lmodern}
\usepackage{hyperref}
\usepackage{subcaption}
\usepackage{placeins}
%\usepackage{amssymb}
\usepackage{fixltx2e}
\usepackage{natbib}
\usepackage[crop=pdfcrop,process=auto]{pstool}
\usepackage{multirow}
\usepackage{enumerate}

\numberwithin{equation}{section}

\newcommand{\Sst}[1]{^\text{\tiny #1}}
\newcommand{\sst}[1]{_\text{\tiny #1}}

\bibliographystyle{plainnat}

\title{}
\author{Matthew Russell}
\date{}

\begin{document}
\maketitle

\section{Gillespie algorithm}
The Gillespie algorithm is a stochastic simulation algorithm, named after
Gillespie due to the papers \cite{gillespie1976general,gillespie1977exact}. It
was described by Gillespie in terms of its application to simulating chemical
reactions, but it is applicable to any stochastic equation (?).
\cite{anderson2007modified} gives a clear description of the algorithm (again in
terms of chemical reactions).

\begin{figure}[ht!]
    \centering
    \psfragfig[width=0.6\textwidth]{figures/urns}
    {
        \psfrag{1}{\(1\)}
        \psfrag{2}{\(2\)}
        \psfrag{3}{\(3\)}
        \psfrag{N-1}{\(N-1\)}
        \psfrag{N}{\(N\)}
    }
    \caption{\label{fig:urns}Set up of the urns containing the particles}
\end{figure}

Suppose we have a physical system with \(N\) possible locations (``urns'') that
each particle can reside in (we impose no upper limit on the number of particles
in each urn), as in figure~\ref{fig:urns}. Denote the number of particles in urn
\(i\) by \(n_i\). Also suppose that there are \(M\)
different events that can occur in the system, such as a particle moving from
one urn to the adjacent urn, or the spontaneous appearance of a particle in the
first urn. These events must occur at specified rates, which represent the
number of times per second one expects the corresponding event to occur.  Given
the rates \(T_i, 1 \le i \le M\) at which the events in the system are expected
to occur, the basic idea of the Gillespie algorithm is to randomly choose the
time increment and then randomly choose the event that will occur at the new
time. The random time increment is drawn from an exponential distribution with
parameter \(T_0 = \sum_{i=1}^M T_i\), the sum of all of the rates. The rates of
the events are not necessarily constants and therefore must be recalculated at
each time step. For example, the rate at which a particle jumps from one urn to
an adjacent urn might depend on the number of particles in the original urn.

The Gillespie algorithm is as follows:
\begin{enumerate}[\bfseries Step 1:]
    \item Set the initial number of particles in each urn and set time \(t=0\)
    \item Calculate the rate \(T_i\) for each of the \(M\) events
    \item Set \(T_0 = \sum_{i=1}^M T_i\)
    \item Generate a uniformly random real number \(r\), such that \(0 \le r \le 1\)
    \item Set \(\delta t = \ln\left(\frac{1}{r}\right)/T_0\) (i.e. an
        \(\delta t\) is drawn from an exponential distribution)
    \item Choose a event randomly from the discrete distribution such that
        probability of drawing event \(k\) is \(T_k/T_0\), for \(1 \le k \le
        M\)
    \item Increment time according to the time step, \(t \leftarrow t + \delta
        t\), and perform the changes corresponding to the event chosen
    \item Return to Step 2, unless the stopping criteria have been met
\end{enumerate}

\section{Solute transport}
Given a system as in figure~\ref{fig:urns} with \(N\) urns, we model the
transport of a solute by advection, diffusion and spontaneous disappearance of a
particle (corresponding to sinks) by the following set of events:

\begin{table}[ht!]
    \centering
    \begin{tabular}{ c | c | c | c }
        Description & Rate & Effect & Count \\ \hline\hline
        \multirow{2}{*}{Hop left} & \multirow{2}{*}{\(T^-_i = a n_{i+1}\)} &
        \(n_{i+1} \leftarrow n_{i+1} - 1\) & \multirow{2}{*}{\(N-1\)} \\
        & & \(n_i \leftarrow n_i + 1\) \\ \hline
        \multirow{2}{*}{Hop right} & \multirow{2}{*}{\(T^+_i = (a+b) n_i\)} &
        \(n_i \leftarrow n_i - 1\) & \multirow{2}{*}{\(N-1\)} \\
        & & \(n_{i+1} \leftarrow n_{i+1} + 1\) \\ \hline
        Inflow & \(T\Sst{in} = c\) & \(n_1 \leftarrow n_1 + 1\) & \(1\) \\ \hline
        Outflow & \(T\Sst{out} = d n_N\) & \(n_N \leftarrow n_N - 1\) & \(1\) \\ \hline
        Removal & \(T\Sst{rem}_i\) = e & \(n_i \leftarrow n_i - 1\) & \(N\) \\
    \end{tabular}
\end{table}

The constants \(a,b,c,d\) and \(e\) are parameters which must be specified, corresponding
to the rates (?) of diffusion, advection, inflow, outflow and removal,
respectively. (Maybe it only makes sense to have \(e=a+b\)?)

There are \(2(N-1) + 2 + N = 3N\) events in total. Note that since
the removal rate is a constant this corresponds to zero-order kinetics, where
the removal rate does not depend on the number of particles in the urn.

\subsection{Example realisations}

\begin{figure}[ht!]
    \centering
    \psfragfig{figures/realisation1}
    {
        \psfrag{t}{\(t\)}
    }

    \caption{\label{fig:exreal1}Realisation of the time evolution of the number
of particles in each urn. \(N=10,a=1,b=0.5,c=1000,d=1.5,e=0\).}
\end{figure}

\FloatBarrier

\bibliography{references}

\end{document}
