\documentclass[a4paper,11pt]{article}

\usepackage[utf8]{inputenc}
%\usepackage[only,llbracket,rrbracket]{stmaryrd}
\usepackage{a4wide}
\usepackage[a4paper]{geometry}
\usepackage{mathtools}
\usepackage[T1]{fontenc}
\usepackage{lmodern}
\usepackage{hyperref}
\usepackage{subcaption}
\usepackage{placeins}
%\usepackage{amssymb}
\usepackage{fixltx2e}
\usepackage{natbib}
\usepackage[crop=pdfcrop,process=auto]{pstool}
\usepackage{multirow}
\usepackage{enumerate}

\numberwithin{equation}{section}

\newcommand{\Sst}[1]{^\text{\tiny #1}}
\newcommand{\sst}[1]{_\text{\tiny #1}}

\newcommand{\diff}[2]{\frac{\mathrm{d} #1}{\mathrm{d} #2}}
\newcommand{\pdiff}[2]{\frac{\partial #1}{\partial #2}}
\newcommand{\V}[1]{\mathbf{#1}}

\bibliographystyle{plainnat}

\title{}
\author{Matthew Russell}
\date{}

\begin{document}
\maketitle

\section{Gillespie algorithm}
The Gillespie algorithm is a stochastic simulation algorithm, named after
Gillespie due to the papers \cite{gillespie1976general,gillespie1977exact}. It
was described by Gillespie in terms of its application to simulating chemical
reactions, but it is applicable to any continuous time Markov process.
\cite{anderson2007modified} gives a clear description of the algorithm (again in
terms of chemical reactions).

\begin{figure}[ht!]
    \centering
    \psfragfig[width=0.6\textwidth]{figures/urns}
    {
        \psfrag{1}{\(1\)}
        \psfrag{2}{\(2\)}
        \psfrag{3}{\(3\)}
        \psfrag{N-1}{\(N-1\)}
        \psfrag{N}{\(N\)}
    }
    \caption{\label{fig:urns}Set up of the urns containing the particles}
\end{figure}

Suppose we have a physical system with \(N\) possible locations (``urns'') that
each particle can reside in (we impose no upper limit on the number of particles
in each urn), as in figure~\ref{fig:urns}. Denote the number of particles in urn
\(i\) by \(n_i\). Also suppose that there are \(M\)
different events that can occur in the system, such as a particle moving from
one urn to the adjacent urn, or the spontaneous appearance of a particle in the
first urn. These events must occur at specified rates, which represent the
number of times per second one expects the corresponding event to occur.  Given
the rates \(T_i, 1 \le i \le M\) at which the events in the system are expected
to occur, the basic idea of the Gillespie algorithm is to randomly choose the
time increment and then randomly choose the event that will occur at the new
time. The random time increment is drawn from an exponential distribution with
parameter \(T_0 = \sum_{i=1}^M T_i\), the sum of all of the rates. The rates of
the events are not necessarily constants and therefore must be recalculated at
each time step. For example, the rate at which a particle jumps from one urn to
an adjacent urn might depend on the number of particles in the original urn.

The Gillespie algorithm is as follows:
\begin{enumerate}[\bfseries Step 1:]
    \item Set the initial number of particles in each urn and set time \(t=0\)
    \item Calculate the rate \(T_i\) for each of the \(M\) events
    \item Set \(T_0 = \sum_{i=1}^M T_i\)
    \item Generate a uniformly random real number \(r\), such that \(0 \le r \le 1\)
    \item Set \(\delta t = \ln\left(\frac{1}{r}\right)/T_0\) (i.e. a \(\delta
        t\) is drawn from an exponential distribution)
    \item Choose a event randomly from the discrete distribution such that
        probability of drawing event \(k\) is \(T_k/T_0\), for \(1 \le k \le
        M\)
    \item Increment time according to the time step, \(t \leftarrow t + \delta
        t\), and perform the changes corresponding to the event chosen
    \item Return to Step 2, unless the stopping criteria have been met
\end{enumerate}

\section{Solute transport}
Given a system as in figure~\ref{fig:urns} with \(N\) urns, we model the
transport of a solute by advection, diffusion and spontaneous disappearance of a
particle (corresponding to sinks) by the following set of events:

\begin{table}[ht!]
    \centering
    \begin{tabular}{ c | c | c | c }
        Description & Rate & Effect & Count \\ \hline\hline
        \multirow{2}{*}{Hop left} & \multirow{2}{*}{\(T^-_i = a n_{i+1}\)} &
        \(n_{i+1} \leftarrow n_{i+1} - 1\) & \multirow{2}{*}{\(N-1\)} \\
        & & \(n_i \leftarrow n_i + 1\) \\ \hline
        \multirow{2}{*}{Hop right} & \multirow{2}{*}{\(T^+_i = (a+b) n_i\)} &
        \(n_i \leftarrow n_i - 1\) & \multirow{2}{*}{\(N-1\)} \\
        & & \(n_{i+1} \leftarrow n_{i+1} + 1\) \\ \hline
        Inflow & \(T\Sst{in} = c\) & \(n_1 \leftarrow n_1 + 1\) & \(1\) \\ \hline
        Outflow & \(T\Sst{out} = d n_N\) & \(n_N \leftarrow n_N - 1\) & \(1\) \\ \hline
        Removal & \(T\Sst{rem}_i = s_i\) & \(n_i \leftarrow n_i - 1\) & \(N\) \\
    \end{tabular}
    \caption{\label{tab:transport_events}Definitions of the different types of
events in the solute transport system}
\end{table}
Figure~\ref{fig:transport_events} gives a diagrammatic representation of the
different event types.

The constants \(a,b,c,d\) and the \(s_i\) are parameters which must be
specified, corresponding to the rates (?) of diffusion, advection, inflow,
outflow and removal (from urn \(i\)), respectively. (Maybe it only makes sense to have
\(d=a+b\)?)

There are \(M=2(N-1) + 2 + N = 3N\) events in total. Note that since
the removal rate is a constant this corresponds to zero-order kinetics, where
the removal rate does not depend on the number of particles in the urn.

\begin{figure}[ht!]
    \centering
    \begin{subfigure}[b]{0.3\textwidth}
        \centering
        \psfragfig{figures/hopleft}
        {
            \psfrag{i}{\(i\)}
            \psfrag{i+1}{\(i+1\)}
        }
        \caption{Hop left}
    \end{subfigure}
    \qquad\qquad
    \begin{subfigure}[b]{0.3\textwidth}
        \centering
        \psfragfig{figures/hopright}
        {
            \psfrag{i}{\(i\)}
            \psfrag{i+1}{\(i+1\)}
        }
        \caption{Hop right}
    \end{subfigure}

    \begin{subfigure}[b]{0.3\textwidth}
        \centering
        \psfragfig{figures/inflow}
        {
            \psfrag{1}{\(1\)}
        }
        \caption{Inflow}
    \end{subfigure}
    \qquad\qquad
    \begin{subfigure}[b]{0.3\textwidth}
        \centering
        \psfragfig{figures/outflow}
        {
            \psfrag{N}{\(N\)}
        }
        \caption{Outflow}
    \end{subfigure}

    \begin{subfigure}[b]{0.3\textwidth}
        \centering
        \psfragfig{figures/removal}
        {
            \psfrag{i}{\(i\)}
        }
        \caption{Removal}
    \end{subfigure}
    \caption{\label{fig:transport_events}Diagrammatic representation of the
different event types in the solute transport system}
\end{figure}

\FloatBarrier

\subsection{Example realisations}

\begin{figure}[ht!]
    \centering
    \psfragfig{figures/realisation1}
    {
        \psfrag{t}{\(t\)}
    }

    \caption{\label{fig:exreal1}Realisation of the time evolution of the number
of particles in each urn. \(N=10,a=1,b=0.5,c=1000,d=1.5,e=0\).}
\end{figure}

\begin{figure}[ht!]
    \centering
    \psfragfig{figures/realisation2}
    {
        \psfrag{t}{\(t\)}
    }

    \caption{\label{fig:exreal2}Realisation of the time evolution of the number
of particles in each urn. \(N=10,a=2,b=0,c=1000,d=2,e=0\).}
\end{figure}

\FloatBarrier

\section{Master equation}
The master equation describes the time evolution of the probability distribution
on the states of a continuous time Markov processes.

%Its general form is
%
%\begin{equation}
%    \label{eqn:master_eqn_general}
%    \diff{P(i,t)}{t} = \sum_j A_{ij} P(j,t),
%\end{equation}
%where \(P(i,t)\) is the probability that the system is in state with label \(i\)
%at time \(t\), and \(A_{ij}\) is a matrix of rate coefficients.

In the case of solute transport, the states of the system consist of the number
of particles contained in each urn at a given time. It is convenient to use a
vector to collect this data:
\begin{equation}
    \label{eqn:state_vector}
    \V{n} = \left(n_1,n_2, \dotsc, n_N\right)^{T}.
\end{equation}
Also, let \(T(\V{n}|\V{n}')\) denote the rate of transition from state \(\V{n}'\)
to state \(\V{n}\). Then the master equation is \citep{mckane2012stochastic}
\begin{equation}
    \label{eqn:master_eqn_rates}
    \diff{P(\V{n},t)}{t} = \sum_{\V{n}' \neq \V{n}} \left[ T(\V{n}|\V{n}')
        P(\V{n}',t) - T(\V{n}'|\V{n}) P(\V{n},t) \right].
\end{equation}

\subsection{Solute transport}
For the above problem of solute transport, the transition rates in
table~\ref{tab:transport_events} can be combined into a single object using the
notation introduced in the previous section:
\begin{equation}
    \label{eqn:}
    T(\V{n}|\V{n}') = \left\{
        \begin{matrix}
            a n'_{i+1} & \text{for} & \V{n} = \V{n}' + \V{e}_i - \V{e}_{i+1},
            i=1,\dotsc,N-1\\
            (a+b) n'_i & \text{for} & \V{n} = \V{n}' - \V{e}_i + \V{e}_{i+1},
            i=1,\dotsc,N-1\\
            c & \text{for} & \V{n} = \V{n}' + \V{e}_1\\
            d & \text{for} & \V{n} = \V{n}' - \V{e}_N\\
            S_i & \text{for} & \V{n} = \V{n}' - \V{e}_i, i=1,\dotsc,N-1\\
            0 & \text{otherwise}\\
        \end{matrix}
        \right.,
\end{equation}
where \(\V{e}_i = (0,\dotsc,0,1,0,\dotsc,0)^T\) is the vector with a \(1\) in
the \(i\)-th entry and a \(0\) in all others. Using this we can write \(\V{n} =
\sum_{i=0}^{N} n_i \V{e}_i\). We will calculate the terms on the right hand side
of the master equation \eqref{eqn:master_eqn_rates} corresponding to each type
of event separately.

\begin{itemize}
    \item Hop left (\(\V{n} = \V{n}' + \V{e}_i - \V{e}_{i+1}\)):
        \begin{equation}
            \label{eqn:trans_me_hl_term}
            \sum_{i=1}^{N-1} \left[a(n_{i+1}+1) P(\V{n} - \V{e}_i +
                \V{e}_{i+1},t) - a n_{i+1} P(\V{n},t) \right]
        \end{equation}
    \item Hop right (\(\V{n} = \V{n}' - \V{e}_i + \V{e}_{i+1}\)):
        \begin{equation}
            \label{eqn:trans_me_hr_term}
            \sum_{i=1}^{N-1} \left[(a+b)(n_i+1) P(\V{n} + \V{e}_i -
                \V{e}_{i+1},t) - (a+b)n_i P(\V{n},t) \right]
        \end{equation}
    \item Inflow (\(\V{n} = \V{n}' + \V{e}_1\)):
        \begin{equation}
            \label{eqn:trans_me_in_term}
            cP(\V{n} - \V{e}_1,t) - c P(\V{n},t)
        \end{equation}
    \item Outflow (\(\V{n} = \V{n}' - \V{e}_N\)):
        \begin{equation}
            \label{eqn:trans_me_out_term}
            dP(\V{n} + \V{e}_N,t) - d P(\V{n},t)
        \end{equation}
    \item Removal (\(\V{n} = \V{n}' - \V{e}_i\)):
        \begin{equation}
            \label{eqn:trans_me_rem_term}
            \sum_{i=1}^N \left[ S_i P(\V{n} + \V{e}_i,t) - S_i P(\V{n},t)
                \right]
        \end{equation}
\end{itemize}

Combining \eqref{eqn:trans_me_hl_term}--\eqref{eqn:trans_me_rem_term}, we
obtain the master equation for this system:
\begin{equation}
    \label{eqn:trans_me}
    \begin{split}
        \diff{P(\V{n},t)}{t} = &
            \sum_{i=1}^{N-1} \left[a(n_{i+1}+1) P(\V{n} - \V{e}_i +
                \V{e}_{i+1},t) - a n_{i+1} P(\V{n},t) \right]\\
            +&\sum_{i=1}^{N-1} \left[(a+b)(n_i+1) P(\V{n} + \V{e}_i -
                \V{e}_{i+1},t) - (a+b)n_i P(\V{n},t) \right]\\
            +&cP(\V{n} - \V{e}_1,t) - c P(\V{n},t)\\
            +&dP(\V{n} + \V{e}_N,t) - d P(\V{n},t)\\
            +&\sum_{i=1}^N \left[ S_i P(\V{n} + \V{e}_i,t) - S_i P(\V{n},t)
                \right].
    \end{split}
\end{equation}

The expected value of the number of particles in urn \(i\) at time \(t\) is
\begin{equation*}
    \langle n_i \rangle(t) = \sum_{j=1}^\infty j P(n_i = j,t),
\end{equation*}
where \(P(n_i = j,t)\) is the marginal distribution of \(n_i\) (is this the
correct way to do it?). The expected value of the entire state \(\V{n}\) at time \(t\) is
\begin{equation*}
    \langle \V{n} \rangle(t) = \sum_{i=1}^N \langle n_i \rangle(t) \V{e}_i
\end{equation*}

\bibliography{references}

\end{document}
